\chapter{Context}
\section{Use case: TU Delft (working title)}
% Matthijs
% Information about location
This projects main area of interest is the campus of the TU Delft. There are more than 20.000 students using the campus on more than 150 hectares. This emphasises even more the magnitude of this project. The network logs the devices connected to the eduroam access points, which implicitly means logging the (approximate) location of the person carrying the device and more information. This tracking data can be used to derive information about the personality of the person carrying the device, such as the distinction between staff and students, based on the tracked locations. Connection to the Wi-Fi eduroam network is free of charge and requires only a NetID, which all students and staff get upon registration at the university. \\\\
It is very important to understand, that 'no data is also data'. This means that a devices that is not being tracked by any access point for a period of time, is either off-campus or disconnected and still on campus. This provides valuable information when researching the movement patterns. This will be further discussed in the \autoref{preprocessing}. \\\\
The eduroam network of the TU Delft campus consists of 1730 access points, distributed over more than 30 buildings. The data is collected for each of the access points over a period of little more than 3 months. The logs are stored in a database on a virtual server, where it is accessible to the three project groups and the Geomatics staff. The data that is collected and the storage in the database is further described in \autoref{datadescription}. \\\\
The department of Facility Management and Real Estate (FMRE) is the main client for the entire Synthesis Project. They would like to know how the campus is being used, what the hotspots on campus and in buildings are, when people travel the most from one building to another and which buildings are most visited.

\section{Previous research: Rhythm of the campus}
% Matthijs
% Summary of their summary
In the fall of 2014, similar research was conducted during another edition of the Geomatics Synthesis Project. The group "Rhythm of the campus" investigated the use of the Library and the Aula of the TU Delft, to gain insight in patterns the use of the facilities of the Library and Aula. This section will give a short summary of their research (\cite{rhythmofthecampus}).\\\\
During the project, the group used passive Wi-Fi monitoring to detect users of the TU Delft Library and the Aula to gain insight in the occupation, in request of FMRE. They used BlueMark sensors at the Library, Aula and 5 other faculties for a period of one week and collected ground truth data for 2 days. Due to its sheer size, the raw data was difficult to process. The data was filtered from static devices and outliers and the data analysis resulted in identification of the occupation of the Library and the Aula. The end results was a dashboard which visualized the sensor network, data analysis and pattern recognition to help the client in the decision making process.\\\\

This research was different from the research conducted in this Synthesis Project, mainly due the larger size of the eduroam network and the ability to track everybody using the Wi-Fi network.

\section{Privacy}
% Balazs

\section{Data validity and accuracy}
% Balazs

\section{Representativeness}
% Xander

\section{Data description and System of APs}\label{datadescription}
% Xander
