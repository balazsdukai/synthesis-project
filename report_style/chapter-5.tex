\chapter{Conclusion}
First of all it can be concluded from the preliminary results that the Wi-Fi network data is suitable, at least to some extent, for retrieving movement patterns of people. Expected patterns such as a movement peak between building during lunch time, and a morning and afternoon peak of people entering and leaving the campus can be clearly distinguished in the data. Similarly aggregated movement on the map shows the expected result that Aula-Library is the most frequently travelled path. More specific patterns between particular buildings and/or during certain time intervals can easily be derived due to the automated workflow. An example of such a specific pattern is that people moving to the aula most often origin from the faculty of Applied Sciences. Furthermore it can be concluded that Aerospace Engineering and to some extent Architecture are rather isolated compared to the other faculties on the campus. Especially when interpreting the result of movement from and to the campus, it should be taken into account that static devices (mainly laptops) are not filtered yet. Disconnecting a laptop for over an hour will currently still be interpreted as a movement away from the campus and back.

