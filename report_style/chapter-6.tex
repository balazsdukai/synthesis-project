\chapter{Recommendations}
Recommendation for future research is first of all the filtered of static devices, as this could improve the quality of the results. Furthermore additional research is required to find methods that can be used for determining the quality of the results, especially concerning the movement from and to the campus.

The detail of the data could be improved by increasing the frequency with which the eduroam system is scanning. This could especially support analysis on the use of different entrances. With the present frequency of one scan round per 5 minutes it is highly likely that the first scan when someone is entering a building is not at the entrance. Detail on which routes are travelled between buildings and validating movement from and to the campus could possible be accomplished by strategically placed scanners outdoor. Finally it is recommended to store the locations of access points digitally in a single map. This would support the process of identification of movement patterns inside buildings. 

Finally caution should be taken with the dataset regarding people their privacy. It is relatively easy to identify a particular user and track that person via the data.
