\chapter{Introduction}\label{Introduction}
\section{Intro}\label{intro}
Wireless Local Area Networks (WLAN) are widely used for indoor positioning of mobile devices within this network. The use of the Wi-Fi network to estimate the location of people is an attractive approach, since Wi-Fi access points (AP) are often available in indoor environments. Furthermore, smart phones are becoming essential in daily life, making it convincing to track mobile devices. This provides a platform to track people by using Wi-Fi monitoring technology. Knowledge of people’s locations and related routine activities are important for numerous activities, such as urban planning, emergency rescue and management of buildings.\\\\
To understand the human motion behaviour many studies are conducted based on data collection of GPS receivers. The Global Navigation Satellite System (GNSS) is commonly used to track people in large scale environments. However due to poor quality of received signals from satellites in urban or indoor environments, GNSS receivers are not suitable in these environments. Moreover, GNSS receivers are convenient for self-tracking, but for large scale movement analysis, this data should be made available first before others can use it. This led to the development of alternative technologies to track people’s locations, including Bluetooth, Dead Reckoning, Radio frequency identification (RFID), ultra-wideband (UWB) and WLAN (\cite{mautz2012indoor}). WLAN has the advantage of widespread deployment, low cost and with the use of a smartphone as a receiver, the possibility to track a large amount of people.\\\\
In general, there are four different location tracking techniques by using the Wi-Fi network: Propagation modelling, multilateration, Fingerprinting and Cell of Origin (CoO). Many of these methods rely on Received Signal Strength Indicators (RSSI) and/or previous set of calibration measurements. In comparison, CoO is the most straightforward technique and uses the location of the AP, to locate the mobile device. For, the location of the AP a mobile device is connected to, will give an estimation of the mobile devices' location, and thus the person. For this project, APs related to buildings and building-parts are used to track people’s movement.\\\\
At the Technical University of Delft (TU Delft) a large scale Wi-Fi network is deployed across all facilities covering the indoor space of the campus. The network is known as an international roaming service for users in educational environments and called the eduroam network. It allows students and staff members from one university to use the infrastructure throughout the campus for free. This allows for large scale collection of Wi-Fi logs including individual scans of mobile devices.  A continuous collection of re-locations of devices to access points for a long duration will return detailed records of people’s movement. This ubiquitous and individual history location data derived from smartphones will present valuable knowledge on movement on the campus. For this reason, the project is carried out in request of the University’s department of Facility Management and Real Estate (FMRE).\\\\
In this project, Wi-Fi monitoring technology is used to discover movement
patterns on the campus of TU Delft. Based on the relationship between activities and places, location history can be used to discover significant places, movement patterns and hotspots. FMRE can use this information to answer questions such "what is the relation between buildings", "where do people come from" and "how regular a trajectory occurs". This project will present a method for identification of movement patterns in a large scale indoor environments and between buildings. The method uses concepts of sequential pattern mining. Previous research has been done on sequential pattern mining, such as \cite{zhao2014discovering} to discover people’s life patterns from mobile Wi-Fi scans, \cite{meneses2012large} analysed place connectivity using the eduroam network and \cite{radaelli2013identifying} identifies indoor movement patterns by analysing a sequence of relocations. Individual movement can be identified as a sequence of relocations of a mobile device to different APs. Without any data between two subsequent re-locations, sequential analysis is a convincing way for identifying moving patterns from wifilogs.
\section{Purpose statement}\label{purpstate}
The project is initiated by the idea that communication technologies can also be used to collect information about connections and connection attempts to Access Points (APs). This geo-referenced information can potentially be used to: \begin {enumerate*} [label=\itshape\arabic*\upshape),font={\color{red!0!black}\bfseries}] \item estimate the number of devices at a location at a certain time, representing presence of people at that place at that time or for a certain duration; \item track unique ID’s over several APs, reconstructing individual patterns of movement, resulting in
aggregated flows of people and; \item define regular and irregular (temporal, deviating) activities at specific places.\end{enumerate*}\\\\
This research will focus on the second matter. Identifying movement patterns has attracted significant interest in recent years. Numerous methods have been explored including Wi-Fi tracking. This report will explain how movement patterns can be identified using large scale Wi-Fi based location data, and tries to contribute with four proposes.
\begin {enumerate*} [label=\itshape\arabic*\upshape),font={\color{red!0!black}\bfseries}] \item A method for identifying movement patterns by analysing individual sequences of relocations from a large scale Wi-Fi network; This includes filtering the raw data and automatically create individual trajectories over a time interval as a sequence of relocations; \item Identify spatio-temporal movement patterns of large crowds of people; \item Investigate different visualization methods for showing movement, based on a large scale Wi-Fi network. \item A method for analysing indoor movement using a constructed network graph of the underlying building floorplan.\end {enumerate*}\\\\
The contributions can be described in one research question for this project.
\begin{itemize}
\item To what extend can movement patterns in and between buildings be identified from large scale Wi-Fi based location data of the eduroam network?
\end{itemize}
In order to answer the research question, there are three applied subquestions:
\begin{itemize}
\item What patterns can be identified moving from and to the TU Delft campus?
\item What movement patterns can be identified between buildings on TU Delft campus?
\item What movement patterns can be identified between large indoor regions of the Faculty of Architecture?
\end{itemize}
Besides looking at this project from a spatial pattern perspective, this research also aims to investigate the following topics:
\begin{itemize}
\item Privacy – how viable is the data for personal concerns?
\item Validity \& Accuracy – how reliable is the data, how accurate, how robust for errors?
\item Representativeness – which amount of the actual users is covered? Is this ratio constant or
location dependant? 
\item System of APs – how well is the system equipped for measuring and tracking, and what is missing /essential to use the system this way?
\end{itemize}

\section{Methods}\label{methods}
The Geomatics Synthesis Project (GSP) is a small research project that combines a literature study with practical research. This includes a case study of the TU Delft campus, using real-world data. Practical work includes data storing, processing, analysing, interpretation, visualization and validation. The project is carried out in a team of six students with a connection to a supervisor and stakeholders (FMRE). This involves interactive discussions between stakeholders as an important part of the research.

\section{Top level requirements}
To keep track of the progress of the project, it is necessary to monitor to which degree the project is meeting the top level requirements and if the project is still on schedule with these requirements. In the baseline review the requirements are specified using the MoSCoW rules and killer requirements. In this chapter these requirements will be discussed.\\\
\textbf{MUST} building level
\begin{itemize}
\item Main goal: Identify movement patterns and connectivity between building entrances.
\item {\color{black!50}Relate entrances (place) of buildings to the corresponding APs (location).}
\item Find the stay places of each individual in order of the scan time.
\item Find individual trajectories from a sequence of stay places.
\item Find the movement patterns, by deriving a sequence of common places shared by all trajectories.
\item Visualize the movement patterns between buildings in static maps.
\end{itemize}
A killer requirement for this level is:
\begin{itemize}
\item {\color{black!50}Identification of APs relating to an entrance of a building}\\\\
\end{itemize}
\textbf{SHOULD} buildingpart level 
\begin{itemize}
\item Main goal: Identify movement patterns between large indoor regions. 
\item Create a network graph from the underlying building floorplan for the analysis, where each region is a node.
\item {\color{black!50}Find the movement trajectories between regions as a sequence of stays.}
\item Find the movement patterns between large indoor regions.
\item Visualize the movement patterns between regions of buildings.
\end{itemize}
The killer requirements for this level are:
\begin{itemize}
\item Digital indoor floorplan of the buildings with classified/named regions (e.g. study rooms, canteen, etc.)
\item Georeferenced building floorplans with APs.\\\\
\end{itemize}
\textbf{COULD} room level
\begin{itemize}
\item {\color{black!50}Main goal: Classification of movement patterns at room level.}
\end{itemize}
The killer requirements for this level are:
\begin{itemize}
\item {\color{black!50}Digital indoor floorplan of the buildings with classified/named rooms (offices, classrooms, project studios, corridors, etc)}
\item {\color{black!50}Location of access points}
\item {\color{black!50}Fingerprinting map}\\\\
\end{itemize}
The following chapters will reflect on these requirements, indicating how successful the project is.
\pagebreak

\section{Reading guide}
This report tries to present our research in 15 chapters. \autoref{context} gives an overview of the project information, including the context, location, privacy issues and data description and representativeness. \autoref{movementpatterns} provides background information on movement patterns including a literature review. \autoref{Methodoloy} describes the methodology of our research. After the methodology \autoref{preprocessing} will describe the pre-processing of the raw Wi-Fi dataset. The identification of movement patterns will be described in the three chapters after pre-processig. \autoref{hernoemen} reports on movements, \autoref{trajectories} will discuss trajectory patterns and \autoref{indoormovement} describes movement indoor. Finally, \autoref{conclusion} and \autoref{recommendations} will conclude the report and provide recommendations.

