\chapter{Recommendations}
\section{Entrances}

\section{Association rules} 
% Balazs

\section{Distinguishing user groups}
% Simon
% People moving in groups

\section{Occupancy}
% Matthijs

\section{AP system}
% Matthijs
% Scan time, also outdoors maybe?
The setup of the system that logs the devices connected to access points is directly connected to the accuracy of the processed data. Currently, the APs register every device that is connected to it and the logging system receives all connected devices approximately every five minutes. Additionally, all access points are located indoors, logging every device carried by people using that building. These two aspects of the AP system limit the accuracy of the processed data and thus the movement patterns that can be derived. \\\\
Because the system logs every connected devices once every five minutes, a device will only be registered if the devices is connected for at least five minutes. This will result in discrepancies in the processed data. Devices and thus people walking by an AP will probably not be registered, for they are not connected to that AP for at least five minutes. This is unfortunate, because a person can travel a rather long distance in five minutes, e.g. making it hard to track people indoors. If the system would be logging every device all the time, irrespective of the time the device is connected, the tracking data would contain every AP that a device would connect to and thus provide much more accurate tracking data. Understandably, logging every user every second would result in huge amounts of data, which would most definitely result in performance issues.\\\\
Secondly, because all scanners are located inside buildings, there is little to no information on people when they move from one building to another. Surely something can be told from the time it takes a device from the last scan in one building, to the first scan in the second building. But for outdoor tracking purposes, this system is limited. From some experiments that were conducted on the TU Delft Campus it can be concluded that a device located outdoors near a building can be detected by APs inside the building, but this depends on the antenna in the devices and the exact location of the device in respect to the AP. If more detailed information about movement outdoors is desired, it would be wise to also include outdoor APs in the system.\\\\
To improve further research, it is recommended to take the system of APs into account before actually conducting the research. If outdoor movement tracking is desired, outdoor APs are required. And if tracking indoors is one of the goals, the frequency of logging should be set to an interval that is in the order of magnitude of 10 seconds to one minute, taking data size in consideration.

\section{Data reasoning}
% Matthijs evt Simon
% Trajectories

\section{Visual exploration}
% Ethan/Matthijs
% What did and didnt work