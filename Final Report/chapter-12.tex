\chapter{Conclusions}\label{conclusion}


First of all it can be concluded from the results, that the Wi-Fi network data is suitable for retrieving movement patterns of people. Expected patterns such as a movement peak between buildings during lunch time, and a morning and afternoon peak of people entering and leaving the campus can be clearly distinguished in the data. Additionally, the data shows actually high movement peaks just before the lectures start. Similarly aggregated movement on the map shows the expected result that Aula-Library is the most frequently travelled path. More specific patterns between particular buildings and/or during certain time intervals can easily be derived due to the automated workflow.\\\\
Furthermore, there is a big difference between static and mobile devices. :ADD MORE TEXT HERE:\\\\
At building level, the results really corresponded with the expectation that there is high amounts of movement in the morning, during lunch and in the afternoon. But unexpected was the accuracy at which people arrive for lectures. Peaks of high movement could be distinguished just before a lecture starts (at times 8:45, 10:45, 13:45 and 15:45). Additionally, most movement between buildings is between the Library and the Aula, which again corresponds to the expectations.\\\\
From the trajectories it can be concluded that :SIMON INSERT TEXT HERE:\\\\
%Most common trajectory: World - 3ME - Aula - Library
Moreover, the indoor movement analysis shows that Architecture people have the tendency to be late at lectures and will leave earlier if it suits them. This could also be explained by the open form of education, because designing at an atelier is not limited to lecture times. At the ground floor level, there seems to be high amount of movement from the restaurant to the orange hall and back.

To conclude the research and answer the research question: Yes the eduroam network at the TU Delft is suitable for finding movement patterns between buildings. Additionally research showed that it is even possible to do small graph analysis on indoor movement. Distinguishing entrances however is not yet possible. The current system set-up proved not suitable, due to the low frequency of logging.