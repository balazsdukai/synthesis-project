\chapter{Trajectory patterns}
\section{Introduction}
This GSP attempts to identify people’s movement patterns from anonymized wifi logs. \autoref{hernoemen} described movement patterns including spatial and temporal aspects of single movements of a crowd of people. Another way of looking at movements, is by tracking individual movement for a longer time interval. A large set of individual trajectories can be used for the identification of typical movements among users of the campus. The method uses concepts from sequential pattern mining. \\\\
This chapter presents a method for identifying movement patterns using individual trajectories. As described in \autoref{movementpatterns}, if moving individuals share some locations in their trajectory, you can speak of co-location in space. When the order of the shared locations are similar for multiple trajectories, you can speak of typical movement. This concept is explored for the identification of movement patterns, and thus the usage of the campus. This approach can answer different questions than looking at single movements, as is done in \autoref{hernoemen}. For example, ‘how many places the user frequently visits’, ‘at what order the user visits places’, ‘how often a trajectory happens’, ‘how many places contained in a frequent trajectory’.\\\\
First, this chapter will describe the problem description, including the extraction of locations of a user, the mining of individual trajectories from an anonymized Wi-Fi scan list, and finaly the mining of movement patterns from a set of trajectories using the PrefixSpan algorithm.
\section{Problem description}
% schrijf ff intro hier
\subsection{Location extraction}
The data provided by the eduroam network enables a detailed view of people’s movement on campus. The large coverage of the eduroam network allows to track users for a large part of the day when they enter the campus. However, the observation space is limited to the extent of the size of the campus, making it not possible to track people outside the eduroam network. A second disadvantage is the spatial resolution of the positioning method. The range a mobile device can be connected to an AP,  influences the accuracy of the estimated location of a mobile device. For indoor environments of the TU Delft campus, this is just a few tens of meters wide. This resolution allows tracking movement at a building level by re-locating mobile devices to the closest AP. Data between two re-locations is not available. Therefore, an individual’s trajectory is depicted by connecting the re-locations as a sequence of APs. These individual trajectories are used to identify patterns. \\\\
A location represents a geographic position where a user stays, i.e. a user is in state. For identifying movement patterns from Wi-Fi monitoring, we are interested in movement between two locations where an individual stays for a longer time period. Such a location, or stay place, can be detected when a user is connected to the same AP for a longer time. To detect  buildings as a location (i.e. contains multiple APs), two consecutive Wi-Fi scans must contain  APs of the same building. With a data collecting interval of 5 minutes, it means that people will be filtered out if their stay duration is less than 10 minutes. Based on this assumption, people with a shorter stay duration are considered passing by, as explained in chapter 8.

\subsection{Individual trajectory}
An individual’s trajectory is constructed as a sequence of locations in order of the scan time. Start and end time of a trajectory can be specified with a time interval. Two consecutive scans from the Wi-Fi log are considered in the same trajectory if and only if \textit{$t_{s2}$ - $t_{e1}$} \textless \textit{$T_{split}$}, where \textit{$T_{split}$} is the splitting threshold. The splitting threshold is important when dealing with people, who are not observed for a long duration of time, i.e. people moving home. For example, if a student leaves the campus at the end of the day, and returns the next morning, seperate trajectories should be created. Because, \textit{$T_{split}$} is larger than the threshold for identifying \textit{'world'} (see \autoref{preprocessing}), the trajectory will always start and end with \textit{'world'}. If \textit{p} is a location, then a trajectory can be written as:
$$p_{1} \rightarrow p_{2} \rightarrow p_{3} \rightarrow …\rightarrow p_{n}$$
Given a time interval, there is a set of individual trajectories $\textit{S} = \{t_{1}, t_{2}, t_{3},...,t_{n}\}$ where each $t_{i}$ is the trajectory. 

\subsection{Trajectory Pattern}
From a set \textit{S} of trajectories, different patterns can be identified using sequential pattern mining algorithms. Frequency of a trajectory by all users of the campus can be detected. This can be represented as a trajectory \textit{T} with a support \textit{s}. Support means how many times the same sequence, or sub-sequence, is shared in the set of trajectories. This gives valuable information on the order common buildings are used and what order of buildings occurs the most. Using a minimum support threshold, sequential mining returns all movement patterns that satisfy \textit{n} \textgreater 2 and support \textit{T} \textgreater \textit{$S_{min}$}   Furthermore, the length of common trajectories can be discovered. This allows for identification of movement patterns of a specific length \textit{n}. Also, when location is not considered, but only the length of a trajectory, the mobility pattern of an individual can be described in terms of how many times he/she re-locates. \\\\
For this study, a trajectory pattern is a sequence of states with \textit{n} \textgreater 2 and support \textgreater \textit{$S_{min}$}. We are only considering trajectory patterns with \textit{n} \textgreater 2, because \autoref{hernoemen} already looked at two consecutive states.\\\\
There exists many developed sequential pattern mining algorithms. For this study PrefixSpan \cite{pei2004mining} is used to identify common shared trajectories or sub-trajectories.  
\section{Implementation}
\section{Results}
% <-- Please divide into the correct sections! --> %





