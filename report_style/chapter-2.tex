\chapter{Top level requirements}
To keep track of the progress of the project, it is necessary to monitor to which degree the project is meeting the top level requirements and if the project is still on schedule with these requirements. In the baseline review the requirements are specified using the MoSCoW rules and killer requirements. In this chapter these previous requirements will be discussed and possible changes will be explained.

The goals that \textit{must} be achieved are on the level of detail of the campus. It’s detailed specification, as stated in the baseline review, is shown below. 

\textbf{MUST} campus level
Main goal: 
\begin{enumerate}
\item Identify which entrances are used to enter and exit a building;
\item Identify movement patterns and connectivity between building entrances by sequential pattern mining.
\end{enumerate}
\begin{itemize}


\item Relate entrances (place) of buildings to the corresponding APs (location).
\item Find the stay places of each individual in order of the scan time.
\item Find individual trajectories by taking a time interval from a sequence of stay places.
\item Find the movement patterns, by deriving a sequence of common places shared by all trajectories.
\item Visualize the movement patterns between buildings in static maps.
\end{itemize}
A \textit{killer requirement} for this level is:

Identification of APs relating to an entrance of a building

Currently the project is progressed so far that it is possible to identify building patterns between buildings. The stay places of individuals and their trajectories have been found and this has been visualized in both static and dynamic maps and bar charts. But, until now there is no accurate map with the location of all access points of the campus. There is such a map for the faculty of architecture, but it is only one building and not very clear. Until this map of the whole campus becomes available, identifying entrances will be hard to do. \autoref{entrances and exists} will go into greater detail about the progress that has been made so far with entrances. 

The goals that \textit {should} be achieved, focus on the building level, where buildings are divided into regions, but since there is currently no map with the locations of the access points, this level of detail is not yet reached. However, the way that the code is setup allows for easy transformation to higher levels of detail when such a map becomes available. How this code exactly works is explained in \autoref{datadescr} and \autoref{movement between buildings}. 
