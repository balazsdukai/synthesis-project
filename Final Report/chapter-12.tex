\chapter{Conclusions}\label{conclusion}
To understand human motion behaviour for better decision making, many studies have been conducted based on location data collection. Wi-Fi tracking technology is increasingly used due its cost effectiveness and ability to track people at a large scale. For this study, we used the eduroam network of the TU Delft Campus to identify movement patterns. Firstly, states are extracted from the raw Wi-Fi logs.  Subsequently, the event of going from one state to another can be detected as movement. Finally, by counting the number of movement for an observation period, movement patterns can be identified. This project tried to illustrate to what extend movement patterns in and between buildings can be identified from large scale Wi-Fi based location data of the eduroam network. In order to answer this question, there are three applied sub questions:
\begin{itemize}
\item What patterns can be identified moving from and to the TU Delft campus?
\item What movement patterns can be identified between buildings on TU Delft campus?
\item What movement patterns can be identified between large indoor regions of the Faculty of Architecture and the built environment?
\end{itemize}
First of all it can be concluded from the results, that the Wi-Fi network data is to some extent suitable for retrieving movement patterns of people. Expected patterns such as a movement peak between buildings during lunch time, and a morning and afternoon peak of people entering and leaving the campus can be clearly distinguished in the data. Additionally, the data shows actually high movement peaks just before the lectures start. Similarly aggregated movement on the map shows the expected result that Aula-Library is the most frequently travelled path. More specific patterns between particular buildings and/or during certain time intervals can easily be derived due to the automated workflow.
\\\\
At building level, the rhythm of the campus is illustrated by time profiles showing the amount of movement for different observation periods. Peaks of high movement could be distinguished just before a lecture starts (at times 8:45, 10:45, 13:45 and 15:45). Additionally, most movement between buildings is between the Library and the Aula, which again corresponds to the expectations.
\\\\
Moreover, the indoor movement analysis shows that Architecture people have the tendency to be a little late at lectures and will leave earlier if it suits them. This could also be explained by the open form of education, because designing in a studio is not limited to strict lecture times. An indoor network graph was created of the underlying building floor plan. This successfully illustrates the occupied space for movement. However, the range of APs can extent between building-parts and floors and limits the accuracy of the analysis.