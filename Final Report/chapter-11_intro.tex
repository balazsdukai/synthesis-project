As described in the firs part of this report, Wi-Fi tracking data can be used to
identify movement between buildings. Given that indoor areas are usually better
covered with Wi-Fi access points than outdoor areas, it is natural to also look
at movement inside buildings. The following section describes our method of
identifying and visualizing indoor movement in the Faculty of Architecture of TU
Delft.

The process of indoor movement analysis is conducted along the steps below,
thus the section also follows this structure:

\begin{enumerate}
    \item Delineate building parts based on the layout of access points
and the division of the building (e.g. department, canteen, building wing), and
group the access point into building parts.
    \item Identify movements in the data between building parts.
    \item Create a route network that connects the building parts and
    is constrained on the corridors of the building.
    \item Assign the movements to the route network.
    \item Visualize the movement along the indoor network.
\end{enumerate}

