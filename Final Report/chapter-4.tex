\chapter{Introduction}\label{introduction}
\section{Intro}\label{intro}
Wireless Local Area Networks (WLAN) are widely used for indoor positioning of mobile devices within this network. The use of the Wi-Fi network to estimate the location of people is an attractive approach, since Wi-Fi access points (AP) are often available in indoor environments. Furthermore, smart phones are becoming essential in daily life, making it convincing to track mobile devices. This provides a platform to track people by using Wi-Fi monitoring technology. Knowledge of people’s locations and related routine activities are important for numerous activities, such as urban planning, emergency rescue and management of buildings.

To understand the human motion behaviour many studies are conducted based on data collection of GPS receivers. The Global Navigation Satellite System (GNSS) is commonly used to track people in large scale environments. However due to poor quality of received signals from satellites in urban or indoor environments, GNSS receivers are not suitable in these environments. This led to the development of alternative technologies to track people’s locations, including Bluetooth, Dead Reckoning, Radio frequency identification (RFID), ultra-wideband (UWB) and WLAN (\cite{mautz2012indoor}). WLAN has the advantage of widespread deployment, low cost and with the use of a smartphone as a receiver, the possibility to track a large amount of people.  

In general, there are four different location tracking techniques by using the Wi-Fi network: Propagation modelling, multilateration, Fingerprinting and Cell of Origin (CoO). Many of these methods rely on Received Signal Strength Indicators (RSSI) and/or previous set of calibration measurements. In comparison, CoO is the most straightforward technique and snaps the location of the mobile device to the same coordinate position as the access point it is connected to. For this project, CoO is used to track people’s movement.

At the Technical University of Delft (TU Delft) a large scale Wi-Fi network is deployed across all facilities covering the indoor space of the campus. The network is known as an international roaming service for users in educational environments and called the eduroam network. It allows students and staff members from one university to use the infrastructure throughout the campus for free. This allows for easy collection of Wi-Fi logs including individual scans of mobile devices.  A continuous collection of re-locations of devices to access points for a long duration will return detailed records of people’s movement. This ubiquitous and individual history location data derived from smartphones will present valuable knowledge on movement on the campus. For this reason, the project is carried out in request of the University’s department of Facility Management and Real Estate (FMRE). 


In this project, Wi-Fi monitoring technology is used to discover movement
patterns on the campus of TU Delft. Based on the relationship between activities and places, location history can be used to discover significant places, movement patterns and hotspots. FMRE can use this information to answer questions such "what is the relation between buildings", "where do people come from" and "how regular a trajectory occurs". This project will present a method for identification of movement patterns in a large scale indoor environments and between buildings. The method uses concepts of sequential pattern mining. Previous research has been done on sequential pattern mining, such as \cite{zhao2014discovering} to discover people’s life patterns from mobile Wi-Fi scans, \citep{meneses2012large} analysed place connectivity using the eduroam network and \cite{radaelli2013identifying} identifies indoor movement patterns by analysing a sequence of relocations. Individual movement can be identified as a sequence of relocations of a mobile device to different APs. Without any data between two subsequent re-locations, sequential analysis is a convincing way for identifying moving patterns from wifilogs.
\section{Purpose statement}\label{purpstate}
Identifying movement patterns has attracted significant interest in recent years. This report will explain how movement patterns can be identified using large scale Wi-Fi based location data. This report tries to contribute with three proposes.
\begin {enumerate*} [label=\itshape\arabic*\upshape),font={\color{red!0!black}\bfseries}] \item A method for identifying movement patterns by analysing individual sequences of relocations from a large scale Wi-Fi network; This includes filtering the raw data and automatically create individual trajectories over a time interval as a sequence of relocations; \item Restructure the association rule mining algorithm to use it in a large scale tracking environment, to discover locations that are commonly associated; \item Investigate different visualization methods for showing movement, based on a large scale Wi-Fi network. \item A method for automatically detect what entrances are used to enter and exit a building.
\end {enumerate*}

The contributions can be described in one research question for this project.
\begin{itemize}
\item[$\textendash$] How can movement patterns be identified from large scale Wi-Fi-based location data of the eduroam network?
\end{itemize}
In order to answer the research question well, there are some sub questions:
\begin{itemize}
\item[$\textendash$] What movement patterns can be identified between buildings on TU Delft campus?
\item[$\textendash$] What movement patterns can be identified between large indoor regions on TU Delft campus?
\item[$\textendash$] What entrances are used to enter and exit a building on TU Delft campus?
\end{itemize}

%% Add the high level/ crosscutting objectives (privacy, validation etc)

\section{Methods}\label{methods}
The Geomatics Synthesis Project (GSP) is a small research project that combines a literature study with practical research. This includes a case study of the TU Delft campus, using real-world data. Practical work includes data storing, processing, analysing, interpretation, visualization and validation. The project is carried out in a team of six students with a connection to a supervisor and stakeholders (FMRE). This involves interactive discussions between stakeholders as an important part of the research. 

\section{Top level requirements}
To keep track of the progress of the project, it is necessary to monitor to which degree the project is meeting the top level requirements and if the project is still on schedule with these requirements. In the baseline review the requirements are specified using the MoSCoW rules and killer requirements. In this chapter these previous requirements will be discussed and possible changes will be explained.

The goals that \textit{must} be achieved are on the level of detail of the campus. It’s detailed specification, as stated in the baseline review, is shown below. 

\textbf{MUST} campus level
Main goal: 
\begin{enumerate}
\item Identify which entrances are used to enter and exit a building;
\item Identify movement patterns and connectivity between building entrances by sequential pattern mining.
\end{enumerate}
\begin{itemize}
\item Relate entrances (place) of buildings to the corresponding APs (location).
\item Find the stay places of each individual in order of the scan time.
\item Find individual trajectories by taking a time interval from a sequence of stay places.
\item Find the movement patterns, by deriving a sequence of common places shared by all trajectories.
\item Visualize the movement patterns between buildings in static maps.
\end{itemize}
A \textit{killer requirement} for this level is:
Identification of APs relating to an entrance of a building

Currently the project is progressed so far that it is possible to identify building patterns between buildings. The stay places of individuals and their trajectories have been found and this has been visualized in both static and dynamic maps and bar charts. But, until now there is no accurate map with the location of all access points of the campus. There is such a map for the faculty of architecture, but it is only one building and not very clear. Until this map of the whole campus becomes available, identifying entrances will be hard to do. \autoref{entrances and exists} will go into greater detail about the progress that has been made so far with entrances. \\\\
The goals that \textit {should} be achieved, focus on the building level, where buildings are divided into regions, but since there is currently no map with the locations of the access points, this level of detail is not yet reached. However, the way that the code is setup allows for easy transformation to higher levels of detail when such a map becomes available. How this code exactly works is explained in \autoref{datadescr} and \autoref{movement between buildings}. 

\section{Reading guide}
% To be written

