\section{Introduction}\label{1-introduction}
% Martijn, Simon
Location is a key element of many processes and activities, and the understanding of human movement behaviour is becoming increasingly important. Knowledge of people’s locations and related mobility patterns are important for numerous activities, such as urban planning, transport planning and facility management. How to efficiently use the available space, is a common problem in many fields. In the educational sector, universities are struggling to meet the higher expectations of facilities for education and research by students and academic staff. Managing the campus of a university has become a complex and challenging task, including the involvement of many stakeholders. Campus managers are in need for evidence-based information to support their decision making (den Heijer, 2012).  This includes better location data to detect activities, occupancy and usage of the infrastructure.
\\\\
To understand the human motion behaviour many studies are conducted based on data collection of GPS receivers. The Global Navigation Satellite System (GNSS) is commonly used to track people in large scale environments. \cite{Stefan} studied the movement of pedestrians in city centres, where potential participants were asked to carry a GPS receiver. However, the distribution of GPS devices to participants limits the possibilities to collect location data at a large scale. Furthermore, due to poor quality of received signals from satellites in indoor environments, GPS receivers are not suitable in these conditions. Technological developments in the acquisition of location data by smart phones and the use of Wi-Fi networks, enables new opportunities to track users. 
\\\\
Wireless Local Area Networks (WLAN) are widely used for indoor positioning of mobile devices within this network. The use of the Wi-Fi network to estimate the location of people is an attractive approach, since Wi-Fi access points (AP) are often available in indoor environments. Furthermore, smart phones are becoming essential in daily life, making it convincing to track mobile devices. This provides a platform to track people by using WLAN as a sensor network, and study the mobility of users inside buildings or groups of buildings. 
\\\\
At Delft University of Technology (TU Delft) a large scale Wi-Fi network is deployed across all facilities covering the indoor space of the campus. The network is known as an international roaming service for users in educational environments and is called the eduroam network. It allows students and staff members from the university to use the infrastructure throughout the campus for free. This enables the possibility to collect Wi-Fi logs, including individual scans of mobile devices, at a large scale.  A continuous collection of re-locations of devices to access points for a long duration will return detailed records of people’s movement. This ubiquitous and individual georeferenced data derived from smartphones will present valuable knowledge about the movement on the campus. 
Several work has been made for studying human mobility patterns in a University’s campus.   \cite{Meneses} used the eduroam network to study connectivity between two places, by computing the number of movements between two places within a given observation time period. Previous work has also been made at TU Delft \cite{rythym}, where several Wi-Fi monitors were placed to detect occupation and movement between different faculties. 
\\\\
In this paper, we attempt to identify people’s movement patterns from the eduroam network of TU Delft. Other than previous studies, this research-driven project analysed data from more than 30.000 users, and tries to detect movement patterns between buildings, and between large indoor regions. The project is carried out in request of the university’s department of Facility Management and Real Estate (FMRE). With this project, we try to illustrate to what extend movement patterns in and between buildings can be identified from anonymised Wi-Fi logs. Firstly, individual states are extracted from the Wi-Fi logs, where users stay for a longer time period. Secondly, movements are detected between a sequence of states. Thirdly, movement patterns can be identified by counting the amount of movement from, to or between certain locations at different time intervals. 
\\\\
The aim of this paper is not to improve a Wi-Fi based positioning technique, but to use the location data to conduct a mobility analysis producing knowledge about the University’s campus. Based on the three steps mentioned above, the aim of this project is to provide a method to detect movement patterns from anonymised Wi-Fi logs. This includes the separation of mobile devices (i.e. smart phones) and static devices (e.g. laptops) from the Wi-Fi logs, and detecting movement to and from beyond the spatial extent of the eduroam network by introducing the concept of a ‘world’ state. Hereby, this paper attempts to contribute with a method to automatically mine people’s movement patterns at two spatial levels. First, movement at building level is analysed. Subsequently, indoor movement at building-part level is studied, by constructing a network graph of the underlying building floorplan. 
The structure of this paper is as follows. In Section 2, describes the case study of TU Delft, the tracking technique and the acquired data that is used in the study. In Section 3 we present our methodology. Section 4 discussed the obtained results. Finally, in Section 5, we present our concluding remarks and recommendations. 

\subsection{Case description}\label{ES-caseDescription}
The project’s main area of interest is the campus of Delft University of Technology (TU Delft), used by more than 30.000 students and staff members. The eduroam network of the TU Delft campus consists of 1730 access points, distributed over more than 30 buildings, covering all indoor space. Even large outdoor areas around the buildings have access to the Wi-Fi network, because of the range of APs. Connection to the Wi-Fi eduroam network is free of charge and requires only a NetID (i.e. username and password), which all students and staff get upon registration at the university. Every time a user accesses the network, the connection is logged. When the connected device moves from one AP to another, a new log is done. The location of the AP a mobile device is connected to, will give an estimation of the mobile devices’ location, and thus the person. This allows the tracking of devices in space and time by relating buildings and building-parts to an aggregation of APs.

The data is collected for every single AP over a period of almost two months. The logs are stored in a database on a virtual server at regular intervals of 5 minutes. In order to ensure privacy, MAC addresses and NetIDs (i.e. usernames) are hashed. Every log is stored with a start time, session duration, AP name and a description of the AP’s location (e.g. System Campus > 20-Aula > 2nd floor). The AP name always contains the ID of the building it is located in. We can use this ID to locate APs at building level. For the Faculty of Architecture and the Built Environment, we also had information about the exact physical position of each AP. This geo-referenced information is used to analyse movement at building-part level.